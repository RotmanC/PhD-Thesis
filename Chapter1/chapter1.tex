%!TEX root = ../thesis.tex
%*******************************************************************************
%*********************************** First Chapter *****************************
%*******************************************************************************

\chapter{Hydrogeological analysis. HYYH and MjPumpit}  %Title of the First Chapter

\ifpdf
    \graphicspath{{Chapter1/Figs/Raster/}{Chapter1/Figs/PDF/}{Chapter1/Figs/}}
\else
    \graphicspath{{Chapter1/Figs/Vector/}{Chapter1/Figs/}}
\fi


%********************************** %First Section  **************************************
\section{Abstract} %Section - 1.1 
%\texorpdfstring{$\sigma$}{[sigma]}
The quantification of the hydraulic parameters is important to support decision making in environmental impact assessment, water resources evaluation or groundwater contamination remediation, among others. These kind of parameters derived from aquifer tests usually encompasses a vast amount of data (spatial and non-spatial) for management and analysis. To achieve this in a clear and understandable manner, the GIS environment is a useful instrument. Development of innovative software to analyse pumping tests in a GIS platform environment to support the hydraulic parameterization of groundwater flow and transport models is presented in this paper. This new platform provides three interconnected modules to improve (a) pumping test interpretation code through a user-friendly interface, (b) pumping test data visualisation supported by a set of tools that perform spatiotemporal queries in a GIS environment and (c) the storage and management of hydrogeological information. Additionally, within the GIS platform, it is possible to process the hydraulic parameters obtained from the pumping test and to create spatial distribution maps, perform geostatistical analysis and export the information to an external software platform. Finally, a real-world application in the area of Barcelona (Spain) has shown the usefulness of the tools developed in support of hydrogeological analysis.

%\begin{align}
%CIF: \hspace*{5mm}F_0^j(a) = \frac{1}{2\pi \iota} \oint_{\gamma} \frac{F_0^j(z)}{z - a} dz
%\end{align}

%\nomenclature[z-cif]{$CIF$}{Cauchy's Integral Formula}                                % %first letter Z is for Acronyms 
%\nomenclature[a-F]{$F$}{complex function}                                                   % first letter A is for Roman symbols
%\nomenclature[g-p]{$\pi$}{ $\simeq 3.14\ldots$}                                             % first letter G is for Greek Symbols
%\nomenclature[g-i]{$\iota$}{unit imaginary number $\sqrt{-1}$}                      % first letter G is for Greek Symbols
%\nomenclature[g-g]{$\gamma$}{a simply closed curve on a complex plane}  % first letter G is for Greek Symbols
%\nomenclature[x-i]{$\oint_\gamma$}{integration around a curve $\gamma$} % first letter X is for Other Symbols
%\nomenclature[r-j]{$j$}{superscript index}                                                       % first letter R is for superscripts
%\nomenclature[s-0]{$0$}{subscript index}                                                        % first letter S is for subscripts


%********************************** %Second Section  *************************************
\section{Introduction} %Section - 1.2

The quantification of hydraulic parameters such as transmissivity (T), hydraulic conductivity (K), storativity (S), and specific storage (Ss) is important for hydrogeological assessments and for the development of groundwater flow and transport models to support decision making in environmental impact assessment, groundwater contamination remediation, water resources evaluation or site monitoring (\citeauthor{Rogiers2012EstimationNetworks}, 2012).
There are several methods to determine hydraulic parameters. The selection of a suitable method depends on the purpose of the research and the required degree of accuracy (\citeauthor{Vukovic1992DeterminationGrain-size}, 1992).  The pumping test is the most commonly used method to obtain aquifer parameters and generally leads to reliable hydraulic parameters. The reliability of the results depends on many factors such as the following: (1) availability of wells and observation points for testing, (2) quality of the pumping test and the subsequent management and analysis of the data obtained and (3) accurate information of aquifer geometry and hydraulic boundaries (\citeauthor{Cheong2008EstimatingKorea}, 2008). Furthermore, a number of specific issues must be considered when dealing with aquifer tests. On the one hand, each problem requires specific solutions (different hydrogeological conditions, different results, different study area and wellbore characteristics, etc.). On the other hand, regarding aquifer test interpretation, a selection of the proper graphical, analytical and numerical solutions for the interpretation should be considered. Moreover, the use of full or partial datasets for the analysis should be taken into account. In this sense, the use of comprehensive tools to store, manage and visualise the vast amount of available data becomes a necessity to focus the analysis and leads to an accurate interpretation.\\
Therefore, it is not surprising that, currently, there is a great amount of software oriented to manipulate and facilitate the calculations of hydraulic parameters from aquifer tests. These computer tools reduce the calculation time and allow a better and more accurate determination of hydraulic parameters. For instance, WTAQ (\citeauthor{Barlow1999WTAQAAquifers}, 1999), WIGAEM (\citeauthor{Bakker2009WigaemFlow}, 2009) or WELLS (\citeauthor{Vesselinov2009AnalyticalAquifer}, 2009) apply analytical solutions for pumping test analysis, but there is still a lack of tools for the pre- and post-processing of input and output data. This is an important problem to be solved because the size of the datasets to be processed is continuously increasing. This growth is closely linked to the importance of water resources, which is becoming more crucial for human communities and because of the greater complexities of the regional groundwater numerical models being used currently. In addition, the hydraulic data available for integration into groundwater numerical models usually has a very diverse origin and format and, therefore, a chance of bias in the interpretations. Consequently, it becomes necessary to have effective instruments that facilitate the pre-process, the visualisation, the analysis and the validation (\textit{e.g.} graphical analysis techniques) of this great amount of data.\\
Software platforms such as MATLAB{\circledR} or MS Excel{\circledR} seem to be more appropriate environments for pre- and post-processing hydrogeological data because of their rich set of tools that are oriented to analyse and visualise data results. Additionally, these software platforms provide a graphical user interface (GUI) for developing specific tools. Some applications oriented to aquifer test data analysis have been developed in the MATLAB{\circledR} environment such as HYTOOL (\citeauthor{Renard2008HytoolGuide}, 2008) or CHOW (\citeauthor{Zhan2001OnAquifers}, 2001), which provide a user-friendly interaction with the scripts for aquifer test analysis. In the same way, MS Excel{\circledR} is a spreadsheet program that allows analysis and plotting of input and output data by using its built-in functions and its Visual Basic for Application (hereinafter VBA) macro option. Some examples of specific tools for pumping test data analysis developed in this context are USGS Spreadsheets by USGS (\citeauthor{HalfordK.J.andKunianksy2002USGSData}, 2002), Molano (\citeyear{Molano2013GroundwaterProblems}) or Johnson and Cosgrove (\citeyear{Johnson2001RADFLOW:ANALYSIS}). In addition, the latter can make draw down contours and mass balance graphs of RADFLOW code.\\
Computer programs that incorporate additional methodologies to process visualise and interpret aquifer test data give one step beyond. Some examples are AQTESTSOLV (\citeauthor{Duffield1989AqTestSolvSolutions}, 1989), Well32 (\citeauthor{GeoSoft2014Well32Geoampsoft}, 1993), AquiferWin32 (\citeauthor{Rumbaugh1997AquiferWin32Analysis}, 1997), Aquifer Test (R\"{o}hrich, \citeyear{Rohrich2002AquiferTestSoftware}) or MLU (\citeauthor{Hemker2010MLUWindows}, 2010). While these software contains many powerful features for performing aquifer tests analysis, further developments in the management of large datasets provided by the aquifer tests and by its interpretations would significantly improve the resulting hydraulic parameterization of the study area. Besides an appropriate storage of all available data and documentation of the procedures used for the interpretation of the pumping tests will facilitate the further updating of the initial parameterization of the hydrogeological model and will also guarantee its future reuse by third parties for different objectives. In this regard, additional advances have been achieved by software such as PIBE (\citeauthor{DiputaciondeAlicante2006PIBEManual}, 2006) or EPHEBO (\citeauthor{Carbonell1997MariaJ_IV:3-D}, 2002).
Despite of these advances, further analysis should be focused on improving the interpretation and validation of the hydraulic parameterization provided by aquifer tests such as the cross-analysis with other datasets (e.g. aquifer geometry or hydraulic boundary conditions). In this sense, the use of a Geographical Information System (GIS) environments represent an optimal solutions for the integration of the aquifer tests data with other relevant datasets (e.g. geological or meteorological datasets) and enable the straightforward queries, search and retrieval of portions of information provided by different sources. Common GIS applications for groundwater research include tools that generate spatiotemporal queries of different hydrogeological parameters. ArcHydro (Maidment, \citeyear{Maidment2002ArcResources}; Strassberg, \citeyear{Strassberg2005ASystems}), CUAHSI (Maidment, \citeyear{Maidment2005CUAHSISymposium}) or GMS (\citeauthor{Jones2004GroundwaterSystem}, 2004) are examples of these types of GIS-based tools, which take advantage of the GIS platforms to manipulate and visualise aquifer information, giving another dimension to the analysis process.\\
By emphasising pumping tests and decision making GIS-based tools, Rios (\citeyear{Rios2011UWATER-PA:Assessment}) developed software (uWATER-PA) for non-specialised issues of groundwater pumping impacts limited by one specific analytical solution for calculations.\\
The foregoing improvements to analyse, visualise and interpret aquifer tests constitute a relevant advantage for the estimation of hydraulic parameters over traditional analysis. However, additional refinements are still needed such as storing, managing and analysing all the available data provided by the aquifer tests interpreted by different methods into a GIS environment. Indeed, the main objective of this work is to provide these needed refinements. To perform that, we provide a package of tools for collecting, managing, analysing, processing and interpreting data derived from pumping tests in a GIS environment. In this sense, the tools developed allow us to apply different methods to interpret aquifer test data in several scenarios taking into account all the information available related with the study zone. Using the standardised spatial database aids to a proper data management. This will increase the understanding and the knowledge in the study area and, thus, will help the development of future projects.\\
The paper is distributed as follows: design and features of the developed software are briefly described in section two. In section three, an application of these tools in a study area is presented and, finally, the conclusions are presented together with additional discussion on the advantages and disadvantages of the software.
%********************************** % Third Section  *************************************
\section{Pumping test software platform} %Section - 1.3 
\label{section1.3}
To facilitate the management and interpretation of pumping tests in a GIS environment, four main requirements should be considered: (a) storing and managing hydrogeological data related to pumping tests in an uniform structure, (b) pumping test data processing and visualisation in a GIS environment, (c) specific tools to interpret or re-interpret pumping tests considering different methods and (d) interoperability with external software for further analysis to complete hydrogeological studies. These requirements mentioned have been reached by the methodology used in the next section.
\subsection{Software design}
To develop the software platform presented in this paper, the following technical criteria were used as guidelines:
\begin{enumerate}
\item A geospatial database to integrate the sets of data from pumping tests. The developed database requires tools and methodologies to simplify the end-user's tasks (user interface and protocols for data exchange).
\item  GIS environment with the following characteristics:
    \begin{enumerate}
    \item  Suite of inherent tools in a GIS environment (\textit{e.g.}, mapping, georeferencing, geostatistical tools, etc.).
    \item  Specialised tools to analyse and integrate hydrogeological data (\textit{e.g.}, to consult points involved in the test, pumping test duration, parameter values estimated, etc.) fed by a structured database.
    \end{enumerate}
\item  Interoperability with external software for further analysis such as aquifer test analysis.
\item  Statistics tools, filters and queries, and different methodologies to solve different aquifer scenarios to guarantee an optimal interpretation.
\item  Post-processing tools. Once the interpretation of pumping tests with external software is performed, the results should be imported again into the GIS platform to continue with the spatial analysis (\textit{e.g.}, useful to prepare maps of parameters to export them to a modelling package). In addition, this analysis can be supported by accurate information on aquifer geometry, hydraulic boundaries (\citeauthor{Cheong2008EstimatingKorea}, 2008) or other spatiotemporal data in a clear and understandable manner.
\end{enumerate}

%\nomenclature[z-DEM]{DEM}{Discrete Element Method}
%\nomenclature[z-FEM]{FEM}{Finite Element Method}
%\nomenclature[z-PFEM]{PFEM}{Particle Finite Element Method}
